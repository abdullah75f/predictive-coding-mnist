\documentclass[a4paper,12pt]{article}
\usepackage{amsmath}
\usepackage{geometry}
\geometry{margin=1in}
\usepackage{parskip}
\usepackage{noto}

\begin{document}

\title{Predictive Coding Model for MNIST Classification}
\author{Abdullah Farid}
\date{June 18, 2025}
\maketitle

\section{Introduction}
Predictive coding is a computational framework inspired by neuroscience, positing that the brain actively predicts sensory inputs rather than passively processing them. It refines its internal model by learning from prediction errors—the mismatch between expectation and reality. This report outlines the successful implementation of a hierarchical predictive coding model for the task of MNIST image classification. The model incorporates both top-down predictions and lateral connections within hidden layers to model feature interactions. The logical flow from neuroscientific theory to a stable, working Python implementation is detailed, covering the model's architecture, its unique learning dynamics, and a final evaluation of its performance.

\section{Predictive Coding Theory}
The core tenet of predictive coding is the minimization of prediction error (or "surprise") throughout a neural hierarchy. The system operates through two distinct but interconnected mechanisms:

\begin{itemize}
    \item \textbf{Inference:} The rapid adjustment of neural activity (\textbf{states}, $x_i$) to find the most likely cause of the current sensory input. This process minimizes immediate prediction error.
    \item \textbf{Learning:} The slow adjustment of synaptic connections (\textbf{weights}, $W_i$) to improve the accuracy of future predictions.
\end{itemize}

In our hierarchical model, a higher layer $i+1$ generates a top-down prediction of the activity in the layer below, $i$. The prediction error $e_i$ is the difference between the actual state $x_i$ and its prediction $\hat{x}_i$. These errors serve as the primary drivers for both inference and learning.

Key equations governing the dynamics are:
\begin{enumerate}
    \item \textbf{Prediction:} The prediction of layer $i$'s state is generated from layer $i+1$: 
    \[ \hat{x}_i = f(W_i x_{i+1} + b_i) \]
    where $f$ is a non-linear activation function (ReLU in our case).

    \item \textbf{Prediction Error:} The error at layer $i$ is the discrepancy between its state and its prediction:
    \[ e_i = x_i - \hat{x}_i \]

    \item \textbf{State Update (Inference):} The state of a hidden layer neuron $x_i$ is updated to reduce error from both above and below. The change in state, $\Delta x_i$, is driven by a bottom-up signal and a top-down signal:
    \[ \Delta x_i \propto (W_{i-1}^T e_{i-1}) - e_i \]
    
    \item \textbf{Weight Update (Learning):} Synaptic weights are updated according to a Hebbian-like rule, based on the final settled states and errors:
    \[ \Delta W_i \propto e_i \cdot f(x_{i+1})^T \]
\end{enumerate}

\section{Model Design and Implementation}
The model is implemented as a Python class, \texttt{PredictiveCodingModel}, using NumPy for computations and PyTorch for the MNIST data pipeline.

\subsection{Architecture}
The network has four layers: an input layer (784 nodes for a flattened 28x28 image), two hidden layers (256 and 64 nodes), and an output layer (10 nodes, one per digit class).
\begin{itemize}
    \item \textbf{Forward Weights ($W_i$):} Connect adjacent layers and are responsible for generating top-down predictions.
    \item \textbf{Lateral Weights ($W_{\text{lateral}}$):} Exist within each hidden layer, allowing neurons to interact and mutually influence their states. This encourages competition and the formation of feature assemblies.
    \item \textbf{States ($x_i$):} Represent the current activation of neurons in each layer.
    \item \textbf{Errors ($e_i$):} Represent the prediction error at each layer.
\end{itemize}

\subsection{The Learning Process: From Theory to Practice}
Translating the theory into a stable learning algorithm required several key implementation details, discovered through iterative development. The training process for a single sample is orchestrated by the \texttt{train()} method and consists of three distinct phases.

\textbf{Phase 0: Providing a Teacher Signal via Clamping}
For supervised classification, the model must be informed of the correct answer. This is achieved by \textbf{clamping} the output layer: the state vector of the final layer, $x_{\text{out}}$, is forcibly set to the one-hot encoded vector of the true label and is held constant throughout the inference phase. This provides a strong, unambiguous error signal that propagates down to the hidden layers.

\textbf{Phase 1: Inference via State Settling}
With the input layer holding the image and the output layer clamped, the model enters an iterative inference loop for a fixed number of steps (e.g., 50 iterations):
\begin{enumerate}
    \item \texttt{forward()}: Top-down predictions are generated at all layers.
    \item \texttt{compute\_errors()}: Prediction errors are calculated. To ensure numerical stability and prevent exploding dynamics, these errors are \textbf{clipped} to a fixed range [-1.0, 1.0].
    \item \texttt{update\_states()}: The states of the hidden layers ($x_1, x_2$) are updated according to the dynamics described in Section 2. The clamped output layer is not updated. This loop allows the hidden states to settle into a configuration that best reconciles the sensory input with the clamped (correct) output.
\end{enumerate}

\textbf{Phase 2: Learning via Weight Updates}
After the inference loop completes, the \texttt{update\_weights()} method is called \textbf{once}.